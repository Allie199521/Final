\documentclass[12pt]{amsart}
\usepackage[top=0.7in,left=0.7in,right=0.8in]{geometry}
\usepackage{color}
\usepackage{mathabx}
\usepackage{enumitem}
\pagestyle{empty}

\begin{document}

\thispagestyle{empty}

\begin{center}
{\bf Cryptography Final - May 14th 2:45pm}\\
{\bf Review Sheet}\\
\end{center}

{\bf Need to know:}\\

\begin{itemize}
\item{} Advantages and Disadvantages of one time pad
	\begin{itemize}
	\item{} {\bf Advantages:}
		\begin{itemize}
		\item{} Impossible to crack if the key is never reused, completely random and kept secret	
		\item{} Immune to brute force attacks - trying all keys simply yields all plaintexts, all equally likely to be the acual plaintext.
		\end{itemize}
	\item{} {\bf Disadvantages:}
		\begin{itemize}
		\item{} Hard to find a truly random key, possible by: psuedorandom number generator
		\item{} Security of the one time pad is only as secure as the exchange of the key - if this is not secure, then the key isn't either.
		\item{} It is difficult to make sure that it continues to remain a secret - dispose of it after first use properly.
		\end{itemize}
	\end{itemize}
\item{} Difference between Stream Ciphers and block ciphers
\begin{itemize}
	\item{} {\bf Block Ciphers:}
	\begin{itemize}
		\item{}More general i.e. can you convert a block cipher into a stream cipher?  Yes, make block size one bit
		\item{}Have no math involved - has to be reversable function
		\item{}Are good in hardware and software but not as good in terms of hardware as stream cipher
	\end{itemize}
	\item{} {\bf Stream Ciphers:}
	\begin{itemize}
		\item{}stream ciphers have more mathematical structure - statistical attacks - easier to break and easier to study
		\item{}stream ciphers are not suitable for software but highly efficient in hardware
	\end{itemize}
	\item{}What is 3DES - three 56 bit keys
	\begin{itemize}
		\item{} Keys to test in worse case $2^{56\cdot 3}$, average $2^{55*3}$
		\item{} 3DES takes in 3 keys, and uses the first key to encrypt a message, the second key to decrypt the encrypted message and then uses the third key to reincrypt the decrypted message.
	\end{itemize}
	\item{}DES - bit length, keys to test in worse case
	\begin{itemize}
		\item{} Keys to test in worse case $2^{56}$, average $2^{55}$
	\end{itemize}
	\item{}Why is 2 DES not secure?
	\begin{itemize}
		\item{} Not secure because the brute force attack of it is less than $2^{90}$.
		\item{} Keys to test in worse case $2^{56\cdot 2}$.
		\item{} 2DES takes 2 keys and encrypts the message with one of them and then decrypts with the other key.
	\end{itemize}
	\item{}What is meet in the middle attack - cuts in half the amount of keys to check
	\item{}BC what is one time pad - attacks on one time pad - use same key - xoring two messages together gets the messages concatenated together.
\end{itemize}
\item{}Brute force attacks and time it will take to do.\\
	\begin{itemize}
	\item{} How to brute force decrypt something.\\
	\end{itemize}
\item{}Most likelyhood of something to happen probability\\
\item{}Factorization of a number made of 2 primes - product of 3 primes instead of 2 primes\\
	\begin{itemize}
	\item{}how to find phi with 3 prime values\\
	\item{}given some cipher from Alice, how would you decrypt it?\\
	\item{}think about it for every algorithm thats out there\\
	\item{}also think about chinese remainder theorem\\
	\end{itemize}
\item{}diffie helman - given $g^a$ and $g^b$, finding $g^{ab}$ is hard... how?\\
	\begin{itemize}
	\item{}given generator, compute the $g^{ab}$\\
	\item{}Elgamal- how it works.\\
	\item{}how to involve 3 people into this?\\
	\item{}sending encrypted message from alice to bob, you have $g^{ab}$ and for bob and carol you get $g^{bc}$.\\
	\item{}m = 59, g = 2, p = 227.  Alice has a = 8, bob b = 6, carol c = 5.  $H_a = 29$, $H_b = 64$, $H_c = 32$ (all mod 227).  Alice will generate $g^{ab}$ using Bobs half mask. $F_{ab} = 12$.  If you don't get the same full mask for bob and alice, its wrong.  Same thing for bob and carol.  $F_{bc} = 44$\\
	\item{}p = 2q+1 - safe prime\
	\item{}q = $\frac{p-1}{2}$\\
	\item{}$g^1 \neq 1$\\
	\item{}$g^2 \neq 1$\\
	\item{}$g^q \neq 1$\\
	\end{itemize}
\item{}Diffie helman - Elliptic Curve\\
	\begin{itemize}
	\item{}Same security in EC - 128 as Elgamal 256.\\
	\item{}Given a curve, only thing on the curve will be the quadratic residues.\\
	\item{}given a set, show me a formula to find the quadratic residues. - Legranges symbol. $(\frac{x}{p}) = x ^{\frac{p-1}{2}} mod\ p$ if we get 1, it is a quadratic residue, -1 is going to be a non quadratic residue.\\
	\item{}finding the square roots of $x$ raise x to the (p+1)/4 and mod by p\\
	\item{}get ascii character to the (x1, y1) character when turning it into a cipher - m is a point on the curve.  ALICE has her own multiplier, bob will have his own multiplier. - use them to encrypt their own half masks B = 4g and A = 3g.  F = B * 3 (bobs halfmask times Alice's multiplier.\\
	\item{}make sure you can find all of the points on the curve. you dont have to find the square roots if the number is not -1 when raised to the power of (p-1)/2 mod the number.
	\item{}the generator value is a point on the curve and the message point is a point on the curve.  ALL OF THE THINGS YOU GET IS A POINT ON THE CURVE.
	\end{itemize}
\end{itemize}


\begin{enumerate}
\item RSA - Public Key Encryption. \\
{\bf Given: }\\ $n$ a small prime\\ $e$ smallest odd integer with gcd with $\phi$ of 1\\ $c$ an encrypted message\\
{\bf Needed: }\\ $p$ and $q$ two prime numbers whose products are n\\ $\phi = (p-1)(q-1)$\\
$d = e^{-1}$\\
\begin{itemize}
\item[(a)] Find the primes $p$ and $q$.  If you do not have a prime factorization on your calculator, then know that one of them is going to be less $\sqrt{n}$, knowing this, we can test all primes less than $\sqrt{n}$.\\
\item[(b)] Calculate $\phi = (p-1)(q-1)$.  From here, it should be easy to find $e$ if it is not given.  Parse through lowest odd values until you find one where $gcd(e, \phi) = 1$.\\
\item[(c)] Now that you have $e$, you have to use pulverizer to solve for $d$. 

\begin{center}
\begin{tabular}{ |cccc|cccc| } 
 \hline
 $\phi$ & $e$ & Quotient & Remainder & $x_1$ & $y_1$ & $x_2$ & $y_2$ \\ 
 \hline
\end{tabular}
\end{center}

\end{itemize}

\end{enumerate}
\end{document}